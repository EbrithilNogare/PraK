%%% Tento soubor obsahuje definice různých užitečných maker a prostředí %%%
%%% Další makra připisujte sem, ať nepřekáží v ostatních souborech.     %%%

%%% Drobné úpravy stylu

% Tato makra přesvědčují mírně ošklivým trikem LaTeX, aby hlavičky kapitol
% sázel příčetněji a nevynechával nad nimi spoustu místa. Směle ignorujte.
\makeatletter
\def\@makechapterhead#1{
  {\parindent \z@ \raggedright \normalfont
   \Huge\bfseries \thechapter. #1
   \par\nobreak
   \vskip 20\p@
}}
\def\@makeschapterhead#1{
  {\parindent \z@ \raggedright \normalfont
   \Huge\bfseries #1
   \par\nobreak
   \vskip 20\p@
}}
\makeatother

% Toto makro definuje kapitolu, která není očíslovaná, ale je uvedena v obsahu.
\def\chapwithtoc#1{
\chapter*{#1}
\addcontentsline{toc}{chapter}{#1}
}

% Trochu volnější nastavení dělení slov, než je default.
\lefthyphenmin=2
\righthyphenmin=2

% Zapne černé "slimáky" na koncích řádků, které přetekly, abychom si
% jich lépe všimli.
\overfullrule=1mm

%%% Makra pro definice, věty, tvrzení, příklady, ... (vyžaduje baliček amsthm)

\theoremstyle{plain}
\newtheorem{veta}{Věta}
\newtheorem{lemma}[veta]{Lemma}
\newtheorem{tvrz}[veta]{Tvrzení}

\theoremstyle{plain}
\newtheorem{definice}{Definice}

\theoremstyle{remark}
\newtheorem*{dusl}{Důsledek}
\newtheorem*{pozn}{Poznámka}
\newtheorem*{prikl}{Příklad}

%%% Prostředí pro důkazy

\newenvironment{dukaz}{
  \par\medskip\noindent
  \textit{Důkaz}.
}{
\newline
\rightline{$\qedsymbol$}
}

%%% Prostředí pro sazbu kódu, případně vstupu/výstupu počítačových
%%% programů. (Vyžaduje balíček fancyvrb -- fancy verbatim.)

\DefineVerbatimEnvironment{code}{Verbatim}{fontsize=\small, frame=single}

%%% Prostor reálných, resp. přirozených čísel
\newcommand{\R}{\mathbb{R}}
\newcommand{\N}{\mathbb{N}}

%%% Užitečné operátory pro statistiku a pravděpodobnost
\DeclareMathOperator{\pr}{\textsf{P}}
\DeclareMathOperator{\E}{\textsf{E}\,}
\DeclareMathOperator{\var}{\textrm{var}}
\DeclareMathOperator{\sd}{\textrm{sd}}

%%% Příkaz pro transpozici vektoru/matice
\newcommand{\T}[1]{#1^\top}

%%% Vychytávky pro matematiku
\newcommand{\goto}{\rightarrow}
\newcommand{\gotop}{\stackrel{P}{\longrightarrow}}
\newcommand{\maon}[1]{o(n^{#1})}
\newcommand{\abs}[1]{\left|{#1}\right|}
\newcommand{\dint}{\int_0^\tau\!\!\int_0^\tau}
\newcommand{\isqr}[1]{\frac{1}{\sqrt{#1}}}

%%% Vychytávky pro tabulky
\newcommand{\pulrad}[1]{\raisebox{1.5ex}[0pt]{#1}}
\newcommand{\mc}[1]{\multicolumn{1}{c}{#1}}


%Define the listing package
\usepackage{color} %use color
\definecolor{mygreen}{rgb}{0,0.6,0}
\definecolor{mygray}{rgb}{0.5,0.5,0.5}
\definecolor{mymauve}{rgb}{0.58,0,0.82}
 
%Customize a bit the look
\lstset{ %
  backgroundcolor=\color{white}, % choose the background color; you must add \usepackage{color} or \usepackage{xcolor}
  basicstyle=\footnotesize, % the size of the fonts that are used for the code
  breakatwhitespace=false, % sets if automatic breaks should only happen at whitespace
  breaklines=true, % sets automatic line breaking
  captionpos=b, % sets the caption-position to bottom
  commentstyle=\color{mygreen}, % comment style
  deletekeywords={...}, % if you want to delete keywords from the given language
  escapeinside={\%*}{*)}, % if you want to add LaTeX within your code
  extendedchars=true, % lets you use non-ASCII characters; for 8-bits encodings only, does not work with UTF-8
  frame=none, % adds a frame around the code
  keepspaces=true, % keeps spaces in text, useful for keeping indentation of code (possibly needs columns=flexible)
  keywordstyle=\color{blue}, % keyword style
  % language=Octave, % the language of the code
  morekeywords={*,...}, % if you want to add more keywords to the set
  numbers=none, % where to put the line-numbers; possible values are (none, left, right)
  numbersep=5pt, % how far the line-numbers are from the code
  numberstyle=\tiny\color{mygray}, % the style that is used for the line-numbers
  rulecolor=\color{black}, % if not set, the frame-color may be changed on line-breaks within not-black text (e.g. comments (green here))
  showspaces=false, % show spaces everywhere adding particular underscores; it overrides 'showstringspaces'
  showstringspaces=false, % underline spaces within strings only
  showtabs=false, % show tabs within strings adding particular underscores
  stepnumber=1, % the step between two line-numbers. If it's 1, each line will be numbered
  stringstyle=\color{mymauve}, % string literal style
  tabsize=4, % sets default tabsize to 4 spaces
  title=\lstname % show the filename of files included with \lstinputlisting; also try caption instead of title
}
%END of listing package%
 
\definecolor{darkgray}{rgb}{.4,.4,.4}
\definecolor{purple}{rgb}{0.65, 0.12, 0.82}
 
%define JavaScript language
\lstdefinelanguage{JavaScript}{
  keywords={then, let, await, break, case, catch, class, const, continue, debugger, default, delete, do, else, enum, export, extends, false, finally, for, function, if, import, in, instanceof, new, null, return, super, switch, this, throw, true, try, typeof, var, void, while, with, yield},
  keywordstyle=\color{blue}\bfseries,
  ndkeywords={JSON, fetch, console},
  ndkeywordstyle=\color{purple}\bfseries,
  identifierstyle=\color{black},
  sensitive=false,
  comment=[l]{//},
  morecomment=[s]{/*}{*/},
  commentstyle=\color{purple}\ttfamily,
  stringstyle=\color{red}\ttfamily,
  morestring=[b]',
  morestring=[b]",
}

\lstset{
  language=JavaScript,
  extendedchars=true,
  basicstyle=\footnotesize\ttfamily,
  showstringspaces=false,
  showspaces=false,
  numberstyle=\footnotesize,
  numbersep=9pt,
  tabsize=2,
  breaklines=true,
  showtabs=false,
  captionpos=b,
  frame=l,
  rulecolor=\color{green!50!black},
  framerule=2.5pt,
}