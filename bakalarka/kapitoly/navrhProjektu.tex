\chapter{Návrh projektu}

\section{Výběr technologií}

\subsection{Frontend}
Ačkoliv se trendy v oblasti vývoje webových aplikací mění velmi rychle,
jedním z nejpodstatnějších trendů, který přenesl vykreslování stránky
na stranu uživatele a tím se výrazně odlišil od dosavadních konceptů
vykreslujících stránku na straně serveru, se stala technologie
\textbf{Single page application}. Celá aplikace pak je v tomto duchu
implementována a přizpůsobena s důrazem na plynulost a rychlost aplikace.

\subsubsection{Single page application}
Single page application je technologie umožnující vykreslení jiné stránky,
bez nutnosti posílaní requestu na server.
Uživatel si při prvním spuštění webu stáhne celý balíček webu a 
při opětovném načtení vetšinou sahá jen do své lokální cache.
JavaScriptová knihovna (v tomto případě React)
poté stránku překresluje při uživatelské interakci.
V případě nutnosti stažení / posílaní dat mezi serverem a uživatelem
(např. editace záznamu, nebo načtení existujícího záznamu)
se volá pouze request k API webové služby a tělo requestu obsahuje pouze
užitečné (ne-redundantní) informace. 


\subsubsection{React}
Knihovna React poskytuje single page application technologii.
Jedná se o dobře udržovanou knihovnu, jež byla vyvinuta Facebookem 
jakožto náhrada zastaralého konceptu renderovaní stránky na serveru.
Díky tomu servery nemusejí ztrácet výkon s každou změnou na stránce a
výkon k renderovaní se bere z PC uživatele.
Jádro této knihovny je velmi dobře optimalizované a poskytuje i řadu
debuggovacích nástrojů, což je pro větší projekty nepostradatelná výhoda.  


\subsubsection{Další možné technologie}
Běžnou praxí vykreslování dynamické stránky je její vykreslení na straně serveru,
jako to má např. velmi populární redakční systém WordPress (jež je psaný v jazyce PHP).
Takovýto system je dobře uživatelsky přivětivý, ale za cenu masivního nárůstu
potřebného serverového výkonu.
V případě implementace knihovního systému by to znamenalo vykreslovat
celou stránku (hlavičku, tělo i zápatí) na serveru,
na druhé straně single page application na serveru nic nevykresluje a
pouze minimalisticky posílá požadované informace.


\subsection{Backend}
Mít single page aplikaci na frontendu znamená, že na backendu musí existovat API,
od kterého bude frontend čerpat data.
Navíc zde potřebujeme i systém pro statické odesílaní balíku celé webové stránky.
V rámci udržitelnosti byl použit stejný jazyk jako na frontendu, JavaScript.
Knihovnou, která by umožňovala komplexní správu requestů a zároveň by byla
i na robustnějších projektech programátorsky přehledná, byla zvolena Express.js,
z důvodů uvedených níže.

\subsubsection{Express.js}
Express.js poskytuje nejen odesílaní statických stránek,
což je potřeba při odesílání balíku s React aplikací, ale umí i
custom requesty, potřebné pro rozmanité API a také odesílaní a
lokální ukládaní statických souborů, jako jsou obrázky a textové nebo pdf dokumenty.

\subsubsection{MongoDB}
MongoDB je databázový systém typu non-SQL.
To znamená, že data neuchovává v tabulkách, ale v tzv. schématech.
To má mnoho výhod, z nichž největší je, že nekompletní záznamy nezabírají
svými nevyplněnými daty místo v DB a ukládá se opravdu jen to, co je potřeba.
Další výhodou je styl ukládaní dat a komunikace s DB.
Databáze si data uchovává ve formátu BSON (binární JSON rozšířený o datové typy).
O data si aplikace žádá pomoci query,
která je zcela odlišná od těch u SQL databází,
primárně se zde neposílá query ve formátu string, ale jako JSON objekt,
díky čemuž např. nenastane známá SQL injection.
Znovu ve formátu JSON poté data vrací aplikaci.

\subsubsection{Další možné technologie}
Díky oddělení frontendu a backendu (na rozdíl např. od WordPressu) je možné
na backend nasadit libovolné technologie, které umí posílat requesty.
Příkladem toho mohou být scripty v jazycích PHP, C\#, Python, nebo Perl.
Vzhledem k potřebě implementace neuronových sítí pro pokročilé vyhledávaní
jsme se rozhodovali mezi dvěma jazyky, které jsou vhodné. Těmito jazyky
s velmi dobrými knihovnami pro práci s neuronovými sítěmi, jsou
Python a JavaScript.

\section{Diagram systému}
\begin{figure}[H]
	\centering
	\includegraphics[angle=90,origin=c,width=\linewidth]{img/diagram.png}
	\caption{Diagram systému}
\end{figure}
