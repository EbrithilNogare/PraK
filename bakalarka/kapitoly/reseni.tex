\chapter{Řešení}


\section{Výsledný web}
Funkční web je možné si zobrazit na adrese \url{http://quest.ms.mff.cuni.cz/prak/homepage}.
Nezapomínejte, že pro zobrazení většiny stránek jsou potřeba vyšší oprávnění,
než má nepřihlášený uživatel.

\section{Uživatelská dokumentace}
\subsection{Návštěvník (nepřihlášený uživatel)}
Při prvním příchodu se návštěvník dostává na hlavní webovou stránku,
na které si může přečíst hlavní informace o projektu nebo
navrhované aktuality.\\
Ačkoliv je obsah stránky převážně v češtině, lze přepnout jazyk ovládacích prvků
pomocí tlačítek \texttt{CZ}, \texttt{DE} a \texttt{EN} pravo nahoře.
Pro zobrazení záznamů slouží ikonka lupy v pravém horním rohu, která sméřuje na
vyhledávací rozhraní ve kterém lze záznamy vyhledat (systém podporuje i regexp výrazy),
setřídit pomocí kliknutí na záhlaví požadovaného sloupce a následně požadovaný záznam i 
zobrazit v jeho detailní podobě.\\
V hlavním menu najdeme i několik odkazů na stránky s relativními informacemi a
rubriky jako jsou \textit{novinky}, nebo \textit{střípky z Krkonoš}.\\
Trochu schované jsou pak odkazy na moduly:
\begin{itemize}
	\item 3D model hologram \\
		\url{http://quest.ms.mff.cuni.cz/prak/modules/hologram/}
	\item 2D mapa Krkonoš 19. století \\
		\url{http://quest.ms.mff.cuni.cz/prak/maps/1.jpg}
\end{itemize}

\subsection{Zadavatel}
Stejně jako uživatel se dokáže dostat k zobrazení detailu záznamu, navíc zde má
ale i tlačítko pro editaci záznamu, jež jej přesune do editačního rozhraní.
Zde může vyplňovat pole a následně záznam uložit. Specifikace atipických polí
je v kapitole 5.3.2.\\
K vytvoření nového záznamu má zadavatel v patičce webu připraveny odkazy do
scény \textit{Zadávátka}, které je identické se vzhledem editačního prostředí.\\
V případě potřeby může zadavatel nahrát soubor na server pomocí opět odkazu v
patičce webu s názvem \textit{Nahrát soubor}. Po kliknutí na tlačítko
s ikonkou šipky se mu zobrazí okno pro zvolení souboru a po jeho vybrání se
soubor odešle na server a uživateli se zobrazí přímý odkaz na nahrátý soubor, který
může začít ihned používat.

\subsection{Redaktor}
Pro přístup do redakčního systému stránky stačí v jejím pravém rohu kliknou 
na ikonku tužky viz \textit{Obr. 5.4} nebo použít odkaz \textit{Redakční systém}
v zápatí stránky, kde se mu zobrazí veškeré stránky webu a odtud si může vybrat,
kterou chce editovat. po kliknutí na ikonku oka stránku zobrazí, případně po
kliknutí na ikonku tužky ji začne editovat. Měnit lze vše kromě již existující
url. Více o redakčním systému v \textit{kapitole 5.3.1 - CMS}


\subsection{Admin}
Veškerá správa uživatelů, jejich informací a oprávnění spadá pod administrační rozhraní,
přístupné přes odkaz \textit{Admin} v zápatí stránky.\\
Nového uživatele lze přidat v horní části administračního rozhraní, kam se
zadají potřebné údaje a uživatel se uloží. Níže pak nalezneme seznam uživatelů a
informace o nich, včetně možnosti změny jejich údajů a nebo smazání jejich účtů.
