\chapter{Moduly}

\section{Přidávání nových modulů}

\section{Aktuálně nasazené moduly}



\subsection{Modul hologram}

\subsubsection{Načítání modulu}
Tento modul obsahuje velkou knihovnu a tudíž není efektivní jej mít v primární aplikaci.
Protože by se tyto knihovny načítaly, i když by uživatel s tímto modulem neměl v plánu pracovat.
Což většinu času nebude chtít a tudíž by to pouze vedlo ke zpomalení systému.
Systémem LAZY načítání je tedy celý modul odříznut a načítá se až explicitně pri jeho použití.

\subsubsection{Grafický engine}
Pro vykreslování 3D modelu na webu slouží WebGL, grafická knihovna podobná staré verzi OpenGL.
Reálně jsou dostupné dvě propracované knihovny pro práci ve WebGL.
ThreeJS, které se zaměřuje na statické vykreslování a předvádění.
BabylonJS, která naopak napodobuje známe programy jako unity a má podporu
fyziky a dalších podknihoven užitečných pro tvorbu her.
Pokud pomineme možnost naprogramovat vše od základu, nejvhodnější knihovnou 
je \textbf{ThreeJS}.\\
Z této knihovny využijeme moduly pro načítání modelu a textury, jejich spárování a
efektu Peppers Ghost, který je implementován pomocí 4 nezávislých kamer.


\subsubsection{Peppers Ghost Effect}
\includegraphics[width=.5\textwidth]{img/Peppers_Ghost.jpg}souce: Wikipedia\\
Jedná se divadelní trik, jež skládá pozorovateli dva obrazy přes sebe a vytváří iluzi
hologramu nebo ducha. Původně byl vyvinut pro divadelní představení, v moderní
době však dokáže simulovat i hologram, který známe ze sci-fi.

\subsubsection{Modely a textury}
Modely jsou z webu https://3dwarehouse.sketchup.com a
textury z webu https://www.textures.com/ a https://texturehaven.com/.\\
V programu blender pak byly tyto textury namapovány na objekt a bylo do nich
zapečeno ambient oclusion (stíny vytvořené stísněným prostorem).
