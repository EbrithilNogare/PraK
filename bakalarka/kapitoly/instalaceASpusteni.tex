\chapter{Instalace a spuštění}

\section{Prerekvizity}
Předpokládáme instalaci na operačním systému Windows 10 s následujícími 
předinstalovanými programy.
\subsection{NodeJS}
JavaScriptový interpret v nejnovější verzi lze stáhnout z oficiálního
webu \url{https://nodejs.org/}.\\
Spolu s nodejs se nám nainstaluje i příkaz \texttt{npm}, ten ihned využijeme
k nainstalování jeho lepšího \uv{dvojčete} a to sice \texttt{pnpm}, pomocí
příkazu
\begin{lstlisting}
	npm install -g pnpm
\end{lstlisting}

\subsection{MongoDB}
Databázový nástroj lze stáhnout ze stránky \url{https://www.mongodb.com},
nebo využít službu \uv{MongoDB Atlas}, což je přednastavená databáze
na serveru Googlu (nebo u Amazonu podle preference).\\
Pro snazší konfiguraci je vhodná aplikace \textit{MongoDBCompass},
poskytující oproti webové aplikaci i příkazový řádek.

\subsection{Nginx}
Webový server lze stáhnout z oficiální stránky vývojářů \url{https://www.nginx.com/}.
Konfigurační soubor pak nalezneme dle cesty instalace:\\
\texttt{.../ChocoInstallFolder/nginx-x.xx.x/conf/nginx.conf}\\
do něj vložíme naši konfiguraci serveru (přizpůsobenou vůči místu instalace)
\begin{lstlisting}[keywordstyle=\ttfamily]
	server {
		listen 80 default_server;
		server_name _;

		# React app & front-end files
		location / {
			root /var/www/naki/frontend/build/;
			rewrite ^ /index.html break;
		}
		location /static/{ root /var/www/naki/frontend/build/; }
		location /images/{ root /var/www/naki/frontend/build/; }
		location /locales/{ root /var/www/naki/frontend/build/; }
		location /public/{ root /var/www/naki/frontend/build/; }

		location /modules/{ alias /var/www/naki/modules/; }
		location /uploads/ { alias /var/www/naki/uploads/; }

		# node API reverse proxy
		location /api/ {
			proxy_pass http://localhost:50081/;
		}
	}
\end{lstlisting}


\subsection{Powershell}
Jelikož výchozí příkazová řádka od Windows má svá léta slávy za sebou, je
vhodnější používat powershell, který se syntaxí více podobá linuxovému shellu.
Pro multiplaformní účely je nejlepší jeho \uv{core} verze, kterou lze stáhnout
z github repozitáře (v README.md je odkaz) \url{https://github.com/PowerShell/PowerShell}.

\subsection{Chocolatey balík}
Pro ty, kteří znají a mají balíčkovací manager \uv{chocolatey} pro Windows, je zde
jednoduchý kód pro rychlou instalaci.
\begin{lstlisting}
	choco install nodejs
	choco install mongodb
	choco install mongodb-compass
	choco install nginx
	choco install powershell-core
	npm install -g pnpm
\end{lstlisting}

\section{Zdrojové soubory}
TODO zveřejnit kód na githubu

\section{Instalace}
Ve složce se zdrojovými soubory spustíme následující příkazy:
\begin{lstlisting}[language=bash]
	#####   backend npm install
	cd backend
	pnpm install
	cd ..

	#####   frontend npm install
	cd frontend
	pnpm install

	#####   frontend build
	npm run build
\end{lstlisting}

