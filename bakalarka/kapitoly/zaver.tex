\chapter*{Závěr}
\noindent
Výsledkem práce je webová aplikace a serverová aplikace poskytující API.
\\


Webová aplikace je podle technologie single-page application,
využívající nejmodernější webové technologie jak na straně uživatele, tak i na straně
serveru. Frontend využívá velmi přívětivý design \textit{Material design} a 
poskytuje uživateli přívětivé prostředí pro práci. Zadavatelé zde naleznou
prostředí pro zadávání nových záznamů nekolika druhů, které pak následně
mohou editovat či mazat. Hotové záznamy může kdokoliv prohledávat a
třídit pomocí několika vestavěných parametrů a následně si požadovaný záznam
plně zobrazit. Web obsahuje i redakční systém, který je spravován redaktory,
kteří přidávají články do rubriky Novinky a editují obsah webu.
\\


Vedle hlavní frontend aplikace je možné si načíst i oddělené moduly.
První modul poskytuje hologram 3D modelu, který se hodí zejména na výstavy.
Druhý modul přináší interaktivní prohlížení 2D map z Krkonoš 19. století.
\\


Serverová aplikace poskytuje nejen samotnou webovou aplikaci, ale i 
moduly nezávislé na hlavní aplikaci. Ty pak komunikují se serverem pomocí API,
které je dobře zdokumentováno a poskytuje tak přístup k datům i pro 
externí programátory a jejich programy.
\\


Beta verze systému byla testována reálnými uživateli a bylo do ní
zadáno přes 35000 záznamů různých typů a velikostí.
\\


Samotný web obsahuje mnoho popularizačních článků, informativních stránek týkajících
se administrativy projektu a novinek (například o plánované výstavě).


\addcontentsline{toc}{chapter}{Závěr}
