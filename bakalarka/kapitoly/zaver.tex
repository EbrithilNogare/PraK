\chapter*{Závěr}
\addcontentsline{toc}{chapter}{Závěr}

Předmětem této práce byl návrh a vývoj softwarového řešení pro tématický
digitální archiv. týkající se sběru, uchování a prezentaci dokumentů z oblasti
Krkonoš. Výsledné softwarové řešení mělo agregovat databázi pro ukládání různých
typů dokumentů (resp. objektů) a web pro prezentaci a úpravu těchto dat. - Přičemž
rozhraní pro popis a evidenci dokumentů měl být navržen
tak, aby byla oddělena vrstva pro běžné prohlížení dokumentů, vkládání záznamů o
dokumentech a vrstva (prostředí) pro správu a editaci těchto záznamů.
Navrhovaný software měl zohledňovat typologii dokumentů a zároveň reflektovat
různé (oborové) normy pro bibliografický popis dokumentů. Při návrhu vkládacího
modulu bylo třeba zohlednit normy knihovnické, muzejní i archivní. V tomto směru
navržený software disponuje velkou inovativností v přístupu k popisu dokumentů.
Zároveň bylo třeba při návrhu software (a jeho jednotlivých modulů) třeba
zohlednit požadavek historiků na vzájemnou provázanost jednotlivých vložených
údajů. - Tj. vytvořit takovou strukturu vkládaných záznamů, aby mezi
bibliografickými údaji a rejstříky bylo možné z různých polí vzájemně odkazovat
na jiná pole, resp. aby bylo možno odkazovat mezi bibliografickým záznamem a
rejstříkem, a taktéž mezi rejstříky navzájem.
\\

Výsledkem je webové rozhraní a serverová aplikace poskytující API pro komunikaci
s jednotlivými moduly výsledného software. V práci je popsána jak implementace
backendu, tak i implementace frontendu, ale i jejich vzájemné provázání
prostřednictvím API. Dále jsou v práci popsány další přídavné moduly k
výslednému software.
\\

Vedle hlavní frontend aplikace je možné si načíst i oddělené moduly.
První modul poskytuje hologram 3D modelu, který se hodí zejména na výstavy.
Druhý modul přináší interaktivní prohlížení 2D map z Krkonoš 19. století.
\\


Serverová aplikace poskytuje nejen samotnou webovou aplikaci, ale i
moduly nezávislé na hlavní aplikaci. Ty pak komunikují se serverem pomocí API,
které je dobře zdokumentováno a poskytuje tak přístup k datům i pro
externí programátory a jejich programy.
\\


Beta verze systému byla testována reálnými uživateli a bylo do ní
zadáno přes 35000 záznamů různých typů a velikostí.
\\

Během vývoje betaverze byl software testován \uv{ostrými} daty. Do systému bylo
uživateli zadáno přes 35 000 záznamů různých typů, velikostí či úrovně
kompletnosti. Na základě poznatků od uživatelů byl software laděn směrem k tomu,
aby rozhraní pro vkládání dat a pro jejich editaci bylo uživatelsky co
nejpřívětivější a co nejintuitivnější.
\\
