\chapter{Analýza a požadavky}
Vybíráme jen z open source produktů, abychom mohli nahlédnout
do jejich kódu a lépe porozumět implementaci jejich komponent.
Lépe se pro takový systém vyvíjejí externí moduly, protože
obvykle mají o dost větší komunitu vývojářů a přispěvatelů.


\section{Požadavky na systém}
Systém bude přístupný jako webová aplikace skrze moderní webové prohlížeče.
Zadavatel bude moci přidávat, prohlížet, měnit a mazat metadata a k nim příslušné indexy.
Redaktor bude moci přidávat a měnit články a novinky na webu.
Uživatel bude moci vyhledat záznam podle několika různých kritérií a poté jej zobrazit,
též bude moci zobrazit příspěvky na webu, včetně novinek.
Externí programátoři budou moci k systému přistupovat přes API a získávat nebo měnit data. 


\section{Pro koho je systém určen}
Systém bude sloužit pro pracovníky historického ústavu jakožto pro zadavatele
a redaktory stránek této aplikace a bude uchovávat historická data z oblasti Krkonoš,
tak aby si je návštěvník případně vědecký pracovník mohl přehledně
zobrazit na jednom místě a případně si data i automaticky stahovat přes připravené API.
Zároveň systém obsahuje interaktivní moduly, jež budou k dispozici
návštěvníkům výstavy Pramenů Krkonoš, jako je hologram 3D modelu nebo
prohlížeč historických map.

\clearpage
\section{Existující produkty}
\subsection{KOHA}
\begin{wrapfigure}[7]{r}{0.25\textwidth}
	\includegraphics[width=\linewidth]{img/KOHA_Logo.png}\\
	\caption[LOGO systému koha ze stránky \url{https://www.mainelibit.org/node/77}]{LOGO systému koha}
\end{wrapfigure}
Domovská stránka aplikace: \url{http://www.koha.cz/}\\
Koha je nejrozšířenější open source knihovní systém s širokou komunitou.
Byla vyvinuta na Novém Zélandu roku 2000.
Ale je stále udržovaná a stále rozšiřovaná
(nejnovější update je z konce roku 2020)
\\
Používá SQL databázi.
Je psaná v Perlu, na frontendu využívající JavaScript (ale není jej tolik).

\subsection{Evergreen}
\begin{wrapfigure}[3]{r}{0.3\textwidth}
	\centering
	\includegraphics[width=\linewidth]{img/Evergreen_Logo.png}\\
	\caption[LOGO systému Evergreen ze stránky \url{https://eg-wiki.osvobozena-knihovna.cz/}]{LOGO systému Evergreen}
\end{wrapfigure}
Domovská stránka aplikace: \\\url{https://eg-wiki.osvobozena-knihovna.cz}\\
Další z řady knihovních systému.
Používán např. Pedagogickou a Teologickou Fakultou v Praze.\\
Používá SQL databázi, vykreslovaní probíhá na serveru a nemá uživatelsky přivětivé prostředí.

\subsection{SLIMS}
\begin{wrapfigure}[5]{r}{0.15\textwidth}
	\includegraphics[width=\linewidth]{img/Slims_Logo.png}\\
	\caption[LOGO systému SLIMS ze stránky \url{https://slims.web.id/}]{LOGO systému SLIMS}
\end{wrapfigure}
Domovská stránka aplikace: \url{https://slims.web.id/}\\
Systém se základní funkcionalitou a přivětivým vzhledem.
Není v češtině. Není primárně určen jako knihovní systém, spíš je to univerzální
systém na cokoliv, proto není až tak efektivní a jeho nastavování
by zabralo mnoho času.

\subsection{Porovnání existujících systémů}
Evergreen a SLIMS jsou systémy, které potřebují silně vyškolenou osobu,
aby se o systém starala, narozdíl od systému KOHA, která je intuitivnější a
pro nové uživatele přívětivější, při zachovaní stejné, možná i lepši funkcionality.\\
