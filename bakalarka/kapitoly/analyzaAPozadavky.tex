\chapter{Analýza a požadavky}
Pro počáteční analýzu jsme vybírali jen z open source produktů, abychom mohli nahlédnout
do jejich kódu a lépe porozumět implementaci jejich komponent.
Lépe se pro takový systém vyvíjejí externí moduly, protože
obvykle mají o dost větší komunitu vývojářů a přispěvatelů.


\section{Požadavky na systém}
Systém bude přístupný jako webová aplikace skrze moderní webové prohlížeče.
Zadavatel bude moci přidávat, prohlížet, měnit a mazat metadata a k nim příslušné indexy.
Redaktor bude moci přidávat a měnit články a novinky na webu.
Uživatel bude moci vyhledat záznam podle několika různých kritérií a poté jej zobrazit,
též bude moci zobrazit příspěvky na webu, včetně novinek.
Externí programátoři budou moci k systému přistupovat přes API a získávat nebo měnit data. 


\section{Pro koho je systém určen}
Systém bude sloužit pro (regionální) historiky (explicitně pro pracovníky 
Historického ústavu AV ČR), jakožto pro zadavatele
a redaktory stránek této aplikace a bude uchovávat historická data z oblasti Krkonoš,
tak aby si je návštěvník či případně vědecký pracovník mohl přehledně
zobrazit na jednom místě a případně si data i automaticky stahovat přes připravené API.
Zároveň systém obsahuje interaktivní moduly, jež budou k dispozici
návštěvníkům plánované výstavy Pramenů Krkonoš (jakožto jednoho z výstupů
projektu), jako je hologram 3D modelu nebo prohlížeč historických map.

\clearpage
\section{Existující systémy}
Vzhledem k tomu, že v oblasti knihovních systémů existuje nepřeberné
množství různých, českých i zahraničních, (open-source) knihovních
systémů, některých i účelově navržených pro různé typy fondů, my jsme
se při analýze zaměřili na tři nejrozšířenější z nich. - Systémy Koha,
SliMS a Evergreen.
Ve všech třech případech se jedná právě o open-source systémy.


\subsection{Koha}
\begin{wrapfigure}[7]{r}{0.25\textwidth}
	\includegraphics[width=\linewidth]{img/Koha_Logo.png}\\
	\caption[logo systému Koha ze stránky \url{https://www.mainelibit.org/node/77}]{logo systému Koha}
\end{wrapfigure}
Domovská stránka aplikace: \url{http://www.Koha.cz/}\\
Koha je nejrozšířenější open source knihovní systém s širokou komunitou.
Koha má několik tisíc instalací v knihovnách různých velikostí a
zaměření v desítkách zemí.
(Tento systém je vhodný i pro použití pro konsorcia knihoven.)
Byla vyvinuta na Novém Zélandu roku 2000.
V roce 2005 došlo k zásadnímu vylepšení systému integrací fulltextového
systému Zebra dánské firmy IndexData.
Ale je stále udržovaná a stále rozšiřovaná
(nejnovější update je z konce roku 2020)
\\
Používá SQL databázi.
Je psaná v Perlu, na frontendu využívající JavaScript (ale není jej tolik).
Jeho nespornou výhodou je ten fakt, že systém má obrovskou (mezinárodní)
komunitu, která systém udržuje a nadále rozvíjí.
A pak také to, že se jedná o cloudový systém.


\subsection{SLIMS}
\begin{wrapfigure}[9]{r}{0.15\textwidth}
	\includegraphics[width=\linewidth]{img/Slims_Logo.png}\\
	\caption[logo systému SLIMS ze stránky \url{https://slims.web.id/}]{logo systému SLIMS}
\end{wrapfigure}
Domovská stránka aplikace: \url{https://slims.web.id/}\\
Systém SliMS je spíše orientován na malé knihovny. (Je hojně používaný v Asii.)
Jeho jedinou aplikací v ČR je systém obecniknihovna.cz. - Jedná se však o
výrazně přepravovanou verzi tohoto systému.
Systém se základní funkcionalitou a přivětivým vzhledem.
Není v češtině. Není primárně určen jako knihovní systém, spíš je to univerzální
systém na cokoliv, proto není až tak efektivní a jeho nastavování
by zabralo mnoho času.


\subsection{Evergreen}
\begin{wrapfigure}[5]{r}{0.3\textwidth}
	\centering
	\includegraphics[width=\linewidth]{img/Evergreen_Logo.png}\\
	\caption[logo systému Evergreen ze stránky \url{https://eg-wiki.osvobozena-knihovna.cz/}]{logo systému Evergreen}
\end{wrapfigure}
Domovská stránka aplikace: \\\url{https://eg-wiki.osvobozena-knihovna.cz}\\
Další z řady otevřených knihovních systémů je Evergreen (zpřístupněný pod
licencí \textit{GNU public licenses content}).
Evergreen byl vyvinut v roce 2006 jako systém pro konsorcium více
než 270 veřejných knihoven amerického státu Georgia.
Poté se rozšířil i do dalších států v rámci USA a do Kanady.
Rozšíření mimo anglicky mluvící státy není tak masivní.
Nejvýrazněji se v rámci Evropy tento systém uplatňuje ve Finsku,
neboť finská národní knihovna se rozhodla systém Evergreen nasadit
jako národní knihovní systém. V rámci ČR lze jako nejvýraznějšího
uživatele tohoto systémů označit pražskou knihovnu JABOK. (Byť systém není
v ČR nějak masivněji rozšířen, přesto Evergreen disponuje českou lokalizací.)
Bohužel jeho nevýhodou je, že jeho funkčnost je zaměřena především na tento
konsorciální použití, ale postupně si nalézá cestu i do knihoven akademických.
Ze zajímavých funkcí, kterými se odlišuje od systému Koha, lze vybrat mimo
jiné zobrazení regálu nebo expertní prohledávání obsahu konkrétního pole.
Systém používá taktéž SQL databázi, vykreslování probíhá na serveru
a nemá uživatelsky přívětivé prostředí.



\subsection{Porovnání existujících systémů}
Co se týče porovnání výše uvedených systémů, lze říct, že Koha a Evergreen
vykazují jisté podobné znaky, co se týče funkcionality a uživatelské podpory v
podobě komunity. Koha však disponuje nepoměrně rozsáhlejší komunitou. Obě
komunity postupují přibližně stejně i při řešení chyb a rozvoji systému. Oba
systémy používají systém pro hlášení chyb, navrhování chybějících funkcí apod.
Systém Koha je plně webový (cloudový), což znamená, že není potřeba instalovat
žádný speciální klient pro uživatele, což usnadňuje nasazení a aktualizace
systému. Naopak součástí Evergreenu je i aplikace, kterou musí mít uživatel
(knihovník) nainstalovanou na svém počítači, aby mohl se systémem pracovat.
Podstatné je, že klient je multiplatformní a může být na rozdíl od komerčních
systémů provozován mimo Windows i v operačních systémech Linux a Mac OSX. Pro
usnadnění přechodu na nové verze nabízí Evergreen možnost automatických
aktualizací klienta. Co se týče správy systému, Evergreen a SliMS jsou systémy,
které potřebují silně vyškolenou osobu, aby se o systém starala, narozdíl od
systému Koha, která je intuitivnější a pro nové uživatele přívětivější, při
zachovaní stejné, možná i lepši funkcionality.
