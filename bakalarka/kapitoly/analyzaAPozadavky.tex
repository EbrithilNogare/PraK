\chapter{Analýza a požadavky}
 Vybíráme jen z open source produktů, abychom mohli nahlédnout
do jejich kódu a lepe porozumět implementaci jejich komponent.
Lépe se pro takový systém vyvíjejí nezávisle moduly a
obvykle mají o dost větší komunitu vývojářů a přispěvatelů.


\section{Požadavky na systém}
Systém bude přístupný jako webová aplikace skrze moderní webové prohlížeče.
Zadavatel bude moci přidávat, prohlížet, měnit a mazat metadata a k nim příslušné indexy.
Redaktor bude moci přidávat a měnit články a novinky na webu.
Uživatel bude moci vyhledat záznam podle několika různých kritérií a poté jej zobrazit,
též bude moci zobrazit příspěvky na webu, včetně novinek.
Externí programátoři budou moci k systému přistupovat přes API a získávat nebo měnit data. 


\section{Pro koho je systém určen}
Systém bude sloužit pro historiky jako úložiště dat a pro veřejnost jako jejich zdroj.


\section{Existující produkty}
\subsection{KOHA}
\includegraphics[width=.25\textwidth]{img/KOHA_Logo.png}\\
url: http://www.koha.cz/\\
Koha je nejrozšířenější open source systém s širokou komunitou.
Byla vyvinuta na Novém Zélandu roku 2000.
Ale je stále udržovaná a stále rozšiřovaná
(nejnovější update je z konce roku 2020)
\\
Používá SQL databázi.
Je psaná v Perlu, na frontendu využívající javascript (ale není jej tolik).

\subsection{Evergreen}
\includegraphics[width=.25\textwidth]{img/Evergreen_Logo.png}\\
url: https://eg-wiki.osvobozena-knihovna.cz\\
Další z řady knihovních systému.
Používán např Pedagogickou a Teologickou Fakultou v Praze.\\
Používá sql databázi, vykreslovaní probíhá na serveru a nemá uživatelsky přivětivé prostředí.

\subsection{SLIMS}
\includegraphics[width=.25\textwidth]{img/Slims_Logo.png}\\
url: https://slims.web.id/\\
Systém se základní funkcionalitou a přivětivým vzhledem.
Není v češtině. Není primárně určen jako knihovní systém, spíš je to univerzální
systém na cokoliv, proto není až tak efektivní a jeho nastavovaní
by zabralo mnoho času.

\subsection{Souhrn}
Evergreen a SLIMS jsou systémy, které potřebují silně vyškolenou osobu,
aby se o systém starala, narozdíl od KOHY, která je intuitivnější a
pro nové uživatele přívětivější. Pri zachovaní stejné, možná i lepši funkcionality.\\
