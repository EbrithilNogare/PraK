\chapter{Úvod}
Knihovní systémy ve své podstatě dlouhodobě uchovávají data o naší historii a je důležité abychom
tyto informace uchovali i pro další generace, ale bohužel tyto systémy stagnují v
zastaralých verzích. Čím méně změn tyto systémy prodělají, tím spíše budou zpětně
kompatibilní a uchovaná data se zachovají v co možná nejméně pozměněné verzi.
Takovýto způsob je mezi programátory docela známý pod příslovím
\uv{dokud to funguje, tak na to nešahej}, což na jednu stranu funguje, ale
je důležité si uvědomit, že změna může přinést
zvýšení efektivnosti pracovníků a pohodlí při práci.
\\

Motivací k napsání této práce byla participace na návrhu řesení
pro projekt NAKI II - Prameny Krkonoš. Vývoj systému evidence,
zpracování a prezentace pramenů k historii a kultuře Krkonoš a
jeho využití ve výzkumu a edukaci. 
\\

Hlavním cílem práce je tedy nashromáždit nejnovější trendy v oblasti webových
aplikací a zakomponovat je do fungující aplikace, principiálně podobné
soudobým knihovním systémům. Nový systém chceme udělat řádově rychlejší,
tak aby si frontend na většinu aktivit vystačil sám a nemusel se dotazovat
na server, což je jeden z typických problému zastaralých systémů.
K datům uloženým v databázi musí přistupovat efektivně, a to přes
dobře zdokumentované API, nasazené na serveru tak, aby zbytečně
neplýtvalo výkonem. Server by tak měl zvládnout pracovat na
aktuálním hardware a zároveň obsluhovat požadavky rychleji a
zvládnout jich větší množství, než soudobé knihovní systémy.
\\

Dalším cílem je zakomponování designového jazyka takového,
který byl vyvinut pro přehlednost dat a efektivitu práce.
Systém s dobrým popisem funkčnosti, který je pro uživatele 
intuitivní a vnese do aplikace přehlednost a nadčasový vzhled.
Uživatel se tak oprostí od jednotvárného a \uv{nahňácaného} vzhledu,
ve kterém se špatně orientuje.
\\

Aby se dal systém dále rozvíjet i mimo hlavní projekt, musí
podporovat systém modulů, které budou moci čerpat společná data přes
API a přidat tak do systému novou funkcionalitu.

