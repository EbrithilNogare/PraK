\chapter{implementace backendu}
Backend je cast softwaru, ktera je umistena na serveru a uzivatel
z ni vidi jen vystup, jez dostane. Slouzi pro prijimani pozadavku od
klienta a odesilani pripadne odpovedi nebo provedeni nejake cinosti bez vystupu. 
\\
Soucasti backendu by melo byt API, ktere slouzi jako toto spojeni mezi
serverem a klientem. Dale casto obsahuje CRON table, ktery periodicky
vykonava nejaky script.

\section{server}
Stroj na kterem bezi aktualni verze systemu bezi pod systemem Linux (presneji Ubuntu).
Na virtual machine poskytnute matematicko-fyzikalni fakultou.
\\
Databaze je pak umistena na serveru u Googlu, kvuli snazsi konfiguraci.
Neni vsak problem kdykoliv tuto DB presunout na stejny stroj, na kterem bezi
zbytek backendu a zvysit tim prodlevu zpusobenou praci s databazi.


\section{knihovny}
Pro rychly zacatek vyvoje byla pouzita knihovna Express.js, ktera
umoznuje praci s http requesty potrebnymi pro fungovani API a to 
bez vetsiho mnozstvi konfigurace ze zacatku.
\\
Pro praci s databazi byla pouzita knihovna mongoose, ktera po rychle konfiguraci
umoznuje praci s databazi typu MongoDB.
\\
Dalsimi mensimi pomocnymy knihovnami, jsou
\textbf{cors} (Cross-Origin Resource Sharing) napomahajici s nastavenim hlavicky u API requestu,
\textbf{md5} pro sifrovani hesel a hashovani dat a nakonec
\textbf{cookie-parser} zjednodusujici praci s cookies.

\subsection{Express.js}


\subsubsection{nadstavba pro nahravani souboru}

\subsection{mongoose}


\section{dokumentace}
/api/documentation
\subsection{vlastni mini knihovna pro dokumentaci}

\section{routes}
\subsection{uzivatel}
\subsection{autorizace}
\subsection{zaznam}
\subsection{stranka}
\subsection{nahravani souboru}

\section{models}
\subsection{zaznam}
\subsection{stranka}
\subsection{uzivatel}
