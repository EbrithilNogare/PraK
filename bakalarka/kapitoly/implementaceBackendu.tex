\chapter{Implementace backendu}
Backend je část softwaru, která je umístěna na serveru a uživatel
z ní vidí jen výstup, jež dostane. Slouží pro přijímaní požadavku od
klienta a odesílaní případné odpovědi nebo provedení nějaké činnosti bez výstupu. 
\\
Součástí backendu by mělo být API, které slouží jako toto spojení mezi
serverem a klientem. Dále často obsahuje CRON tabulku, která periodicky
vykonává nějaký script.

\section{Server}
Stroj na kterém běží aktuální verze systému běží pod systémem Linux (přesněji Ubuntu).
Na virtual machine poskytnuté matematicko-fyzikální fakultou UK.
\\
Databáze je pak umístěna na serveru u Googlu, kvůli snazší konfiguraci.
Není však problém kdykoliv tuto DB přesunout na stejný stroj, na kterém běží
zbytek backendu a snížit tím prodlevu způsobenou prací s databází.


\section{Knihovny}
Pro rychlý začátek vývoje byla použita knihovna Express.js, která
umožňuje práci s http requesty potřebnými pro fungovaní API a to 
bez většího množství konfigurace ze začátku.
\\
Pro práci s databází byla použita knihovna mongoose, která po rychlé konfiguraci
umožňuje práci s databází typu MongoDB.
\\
Dalšími menšími pomocnými knihovnami, jsou
\textbf{cors} (Cross-Origin Resource Sharing) napomáhající s nastavením hlavičky u API requestu,
\textbf{md5} pro šifrovaní hesel a hašovaní dat a nakonec
\textbf{cookie-parser} zjednodušující práci s cookies.

\subsection{Knihovna express.js}
Jedná se o minimalistickou a zároveň velmi silnou knihovnu poskytující všestranné prostředky pro web.
Obsahuje funkce pro jednoduchou správu HTTP metod a díky tomu je vytváření i větších API jednoduché a přehledné.
Má velmi dobrý výkon, který se sice nedá srovnávat s knihovnami psanými v jazyce C, ale
z JS knihoven je jeden z nejrychlejších.

\subsubsection{Nadstavba pro nahravaní souborů}
Nadstavba \textbf{express-fileupload} umožňuje v těle requestu rozeznat a zpracovat soubor, který
se následně pomocí knihovny \textbf{fs} ukládá na server, specificky do složky uploads.

\subsection{Knihovna mongoose}
Ideální knihovna pro práci s databází typu MongoDB.
Poskytuje funkce pro snadné připojení k databázi a komunikaci s ní.
Hlavní výhodou této knihovny jsou modely, validace a vestavěné přetypování
(javascript je dynamicky typovaný jazyk). 
S modely se pracuje pomocí tkz. \textbf{promise}, která umožňuje řadit
akce za sebe, ve stylu:
\begin{lstlisting}[language=JavaScript]
Model.find(body)
     .limit(_limit || 5)
     .exec()
     .then(result => { res.status(200).json(result) })
     .catch(err => { res.status(500).json("something went wrong") })
\end{lstlisting}

\section{Dokumentace API}
Na adrese http://quest.ms.mff.cuni.cz/prak/api/documentation se nachází statický soubor s dokumentací.
Celá stránka je zapouzdřena do jediného souboru, který se po načtení 
vykreslí na straně uživatele, tudíž nezatěžuje server, ale
především se dá stáhnout a prohlížet offline.

\subsection{Vykreslovací engine pro dokumentaci}
Aby bylo možné vykreslit stránku až na straně uživatele,
je nutné přenášet i script, který to obstará. Tento script se skládá ze dvou částí.
Engine na vykreslení a data samotná, která jsou uložena ve formátu JSON.\\
Script projde veškerá data a podle typu a kontextu postupně vytváří
html elemety podle vestavěných šablon a přidává jim příslušné styly a ovládací prvky.


\section{Routes}
Pro rozpoznání, která akce se má vykonat při rúzných dotazech, se porovnává
jak adresa dotazu tak i metoda. Nejdříve se vezme v potaz cesta dotazu za
statickou předponou \uv{http://quest.ms.mff.cuni.cz/prak/api/...}.
Jakmile máme vubranou cestu zjistí se, co vlastně dotaz potřebuje udělat a to jednou z těcho metod:
\begin{itemize}
     \item POST - většinou obecný dotaz s query v těle
     \item GET - dotaz požadující 1 záznam, většinou podle ID
     \item PUT - vytvoření nového záznamu
     \item PATCH - změna v existujícím záznamu
     \item DELTE - smazání záznamu
\end{itemize}
Pokud uživatel má požadované oprávnění, akce se provede. V každém případě
uživateli přijde zpětná vazba o úspěchu resp. neúspěchu dotazu. Tělo
této zprávy může obsahovat požadovaná data, potvrzení o úspěchu, nebo i
chybová zpráva a její příčina.

\subsection{Uživatel}
Slouží pro správu uživatelských účtů.
Zakládání nového účtu může zavolat kdokoliv, avšak na zbylé akce, jako
mazání, nebo změnu hesla má právo pouze majitel a administrátor.
Stejně tak heslo a sessionID nejsou přístupné zvenčí.
\subsection{Autorizace}
Pro ověření, zda přihlášený uživatel má příslušná práva
(a to na straně serveru, jelikož lokálně si je může libovoně upravit) se používá
\textbf{sessionID}, které se poté porovnává v databázi s uživatelem a jeho skutečnými právy.\\
SessionID získáme po odeslání korektních přihlašovacích údajů. Případně jej ztratíme při odhlášení, nebo
expiraci (která je aktuálně nastavena na 1 rok).
\subsection{Záznam}
Pro prohlížení záznamu, resp. jejich vyhledávání není třeba žádné oprávnění. Avšak pro jejich editaci resp. mazání
je potřeba mít přiřazena práva pro zápis.
\subsection{Stránka}
Zobrazení stránky též nevyžaduje žádná práva, ale k jejich vytváření je třeba mít roli tkz. \uv{CMS editor}, která
jak název napovídá definuje uživatele jakožto editora článků a příspěvků.
\subsection{Nahrávaní souborů}
Pro nahrávání souborů je třeba práv pro zápis, stejně jako pro mazání.
Zobrazení pak žádná práva nevyžaduje, ani nevyžaduje přístup ze stejné domény.

\section{Modely}
Model nebo též schéma je popis záznamu v databázi, obsahuje datový typ, formát a může obsahovat i
referenční cestu. Vpodstatě se jedná o převodní tabulku, aby se Javascript a MongoDB schodli na datovém typu a
struktuře (především pro případ reference, nebo pole dat). Část která je pro uživatele nejvíce viditelná je 
požadavek na unikátní hodnotu pole, nebo požadavek aby hodnota byla nenulová.
\subsection{Záznam}
Každý záznam má přesný popis v online dokumentaci API v pravém sloupci, kde je pravidelně aktualizován.
\subsection{Stránka}
Záznam stránky obsahuje název, jazyk, krátký popis, kategorii a obsah samotný.
Pro možnost sledování změn je zde i automaticky generovaný seznam editorů a časů jejich editace.
\subsection{Uživatel}
