\chapter{Implementace backendu}
Backend je software, který je umístěn na serveru a uživatel má přístup jen
k výstupu, jež dostane. Slouží pro přijímaní požadavků od klienta a odesílaní
případné odpovědi nebo provedení nějaké činnosti bez výstupu.
\\
Součástí backendu je i API, které slouží jako toto spojení mezi serverem a
klientem. Dále často obsahuje CRON tabulku, která periodicky vykonává nějaký
script.

\section{Server}
Stroj, na kterém je dostupná aktuální verze systému, virtuální stroj poskytnutý
Matematicko-fyzikální fakultou UK, s operačním systémem Linux (přesněji Ubuntu
20.04.2 LTS).
\\
Databáze je pak umístěna na serveru Google kvůli snazší konfiguraci. Není však
problém kdykoliv tuto DB přesunout na stejný stroj, na kterém běží zbytek
backendu, a snížit tím prodlevu způsobenou prací s databází.


\section{Knihovny}
Pro rychlý začátek vývoje byla použita knihovna \textbf{Express.js}, která
umožňuje práci s http requesty potřebnými pro fungovaní API, a to 
bez většího množství konfigurace ze začátku.
\\
Pro práci s databází byla použita knihovna \textbf{mongoose}, která po rychlé konfiguraci
umožňuje práci s databází typu MongoDB.
\\
Dalšími menšími pomocnými knihovnami jsou
\textbf{cors} (Cross-Origin Resource Sharing) napomáhající s nastavením hlavičky u API requestu,
\textbf{md5} pro šifrovaní hesel a hašovaní dat a nakonec
\textbf{cookie-parser} zjednodušující práci s cookies.

\subsection{Knihovna express.js}
Jedná se o minimalistickou a zároveň velmi silnou knihovnu poskytující
všestranné prostředky pro web. Obsahuje funkce pro jednoduchou správu HTTP metod
a díky tomu je vytváření i větších API jednoduché a přehledné. Má velmi dobrý
výkon, který se sice nedá srovnávat s knihovnami psanými v jazyce C, ale z JS
knihoven je jeden z nejrychlejších.

\subsubsection{Nadstavba pro nahrávání souborů}
Nadstavba \textbf{express-fileupload} umožňuje v těle requestu rozeznat a
zpracovat soubor, který se následně pomocí knihovny \textbf{fs} ukládá na
server, specificky do složky uploads.

\subsection{Knihovna mongoose}
Tato knihovna je ideální a nepostradatelná pro práci s databází typu MongoDB.
Poskytuje funkce pro snadné připojení k databázi a komunikaci s ní.
Hlavní výhodou této knihovny jsou modely, validace a vestavěné přetypování
(JavaScript je dynamicky typovaný jazyk). 
S modely se pracuje pomocí tzv. \textbf{promise}, která umožňuje řadit
akce za sebe, ve stylu:
\begin{lstlisting}[language=JavaScript]
Model.find(body)
     .limit(_limit || 5)
     .exec()
     .then(result => { res.status(200).json(result) })
     .catch(err => { res.status(500).json("something went wrong") })
\end{lstlisting}

\section{Dokumentace API}
Na adrese \url{http://quest.ms.mff.cuni.cz/prak/api/documentation}
se nachází statický soubor s dokumentací.
Celá stránka je zapouzdřena do jediného souboru, který se po načtení 
vykreslí na straně uživatele, tudíž nezatěžuje server, ale
především se dá stáhnout a prohlížet offline.

\subsection{Vykreslovací engine pro dokumentaci}
Aby bylo možné vykreslit stránku až na straně uživatele,
je nutné přenášet i script, který to obstará. Tento script se skládá ze dvou částí. -
Engine na vykreslení a data samotná, která jsou uložena ve formátu JSON.\\
Script projde veškerá data a podle typu a kontextu postupně vytváří
html elemety podle vestavěných šablon a přidává jim příslušné styly a ovládací prvky.


\section[Routes]{Routes (Směrovače)}
Pro rozpoznání, která akce se má vykonat při různých dotazech, se porovnává
jak adresa, tak metoda. Nejdříve se vezme v potaz cesta dotazu za
statickou předponou \texttt{http://quest.ms.mff.cuni.cz/prak/api/\dots}.
Jakmile máme vybranou cestu, zjistí se, co vlastně dotaz potřebuje udělat, a to jednou z těchto metod:
\begin{itemize}
     \item POST - většinou obecný dotaz s query v těle
     \item GET - dotaz požadující 1 záznam, většinou podle ID
     \item PUT - vytvoření nového záznamu
     \item PATCH - změna v existujícím záznamu
     \item DELETE - smazání záznamu
\end{itemize}
Pokud uživatel má požadované oprávnění, akce se provede. V každém případě
uživateli přijde zpětná vazba o úspěchu, resp. neúspěchu, dotazu. Tělo
této zprávy může obsahovat požadovaná data, potvrzení o úspěchu, nebo
chybovou zprávu a její příčinu.


\subsection{Uživatel}
Tato metoda slouží pro administraci uživatelských účtů pro tuto aplikaci.
Zakládání nového účtu může zavolat kdokoliv, avšak na zbylé akce, jako
mazání nebo změnu hesla, má právo pouze majitel účtu a administrátor.
Velkou bezpečnostní výhodou je naprostá nepřístupnost k heslu a
přístupnost k sessionID jen pro administrátora nebo vlastníka.

\subsection{Autorizace}
Pro ověření, zda přihlášený uživatel má příslušná práva
(a to na straně serveru, jelikož lokálně si je může libovolně upravit), se používá
\textbf{sessionID}, které se poté porovnává v databázi s uživatelem a jeho skutečnými právy.\\
SessionID získáme po odeslání korektních přihlašovacích údajů.
Případně jej ztratíme při odhlášení nebo
expiraci (která je aktuálně nastavena na 1 rok).

\subsection{Záznam}
Pro prohlížení záznamu, resp. jejich vyhledávání, není třeba žádné oprávnění.
Avšak pro jejich editaci, resp. mazání,
je potřeba mít přiřazena práva pro zápis.

\subsection{Stránka}
Zobrazení stránky též nevyžaduje žádná práva, ale k jejich vytváření je třeba mít roli CMS editora, která
definuje uživatele jakožto editora článků a příspěvků.

\subsection{Nahrávání souborů}
Pro nahrávání souborů na server je třeba práv pro zápis, stejně jako pro mazání.
Zobrazení souboru žádná práva nevyžaduje, dokonce nenastavuje ani cross-origin policy, neboli
nevyžaduje přístup ze stejné domény.

\section{Modely}
Model nebo též schéma je popis záznamu v databázi. Obsahuje datový typ, formát a může obsahovat i
referenční cestu. V podstatě se jedná o převodní tabulku, aby se JavaScript a MongoDB shodli na datovém typu a
struktuře (především pro případ reference nebo pole dat). Část, která je pro uživatele nejvíce viditelná, je 
požadavek na unikátní hodnotu pole nebo požadavek, aby hodnota byla nenulová.
\subsection{Záznam} 
Každý záznam má přesný popis v online dokumentaci API, přesněji v pravé části, kde je pravidelně aktualizován.
\subsection{Stránka}
Záznam stránky obsahuje název, jazyka, krátký popis obsahu, kategorii a obsah samotný.
Pro možnost sledování změn je zde i automaticky generovaný seznam editorů a časů jejich editace.
\subsection{Uživatel}
Každý uživatel se přihlašuje e-mailem a heslem, s tím, že heslo není uloženo v
tzv. raw formátu, ale je výsledkem spojení hesla a náhodného textu pro zvýšení
bezpečnosti (sůl, anglicky \textit{salt}). Výsledný hash (metodou \textit{md5}) se uloží
do databáze. Při přihlášení pak stačí porovnat hash soli a hesla s údaji v
databázi. Pro ověřování práv neslouží heslo, ale sessionID. To se generuje
unikátní pro každého uživatele při přihlášení, ale časem, nebo manuálně, může
expirovat a poté je nutné se přihlásit znovu. V záznamu uživatele je uloženo i
jméno a příjmení pro případ, že by bylo potřeba zjistit, kdo např. psal jaký
článek, a zároveň nebyl prozrazen e-mail. Každý uživatel má určitá práva a to v
libovolné kombinaci z následujících:
\begin{itemize}
     \item READ - právo pro čtení, nahlížení do záznamů a jejich vyhledávaní (toto právo má i nepřihlášený uživatel)
     \item WRITE - právo pro zápis v rejstřících, umožňuje editovat, vytvářet a mazat záznamy. Dále má navíc právo ukládat na server obrázky a dokumenty
     \item EXECUTE - administrátorské oprávnění, umožňuje vytváření uživatelských účtů a změnu hesla uživatele, stejně tak jeho údaje
     \item CMS - práva pro editační změny, co se týče obsahu stránek, jako jsou novinky a příspěvky
\end{itemize}
