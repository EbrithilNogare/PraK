\chapter{existujici produkty}
Vybirame jen z open source produktu, abychom mohli nahlednout
do jejich kodu a lepe porozumet implementaci jejich komponent.
Lepe se pro takovy system vyvyjeji nezavisle moduly a
obvikle maji o dost vetsi komunitu vyvojaru a prispevatelu.


\section{KOHA}
url: http://www.koha.cz/\\
Koha je nejrozsirenejsi open source system s sirokou komunitou.
Byla vyvynuta na Novem Zelande roku 2000.
Ale je stale udrzovana a stale rozsirovana
(nejnovejsi update je z konce roku 2020)
\\
Pouziva SQL tabulky.
Psana v Perlu, na frontendu vyuzivajici javascript (ale neni ho tolik).


\section{Evergreen}
url: https://eg-wiki.osvobozena-knihovna.cz\\
Dalsi z rady knihovnich systemu.
Pouzivan napr Pedagogickou a Teologickou Fakultou v Praze.\\
Pouziva sql databazi, vykreslovani na serveru, nema uzivatelsky privetive prostredi.



\section{SLIMS}
url: https://slims.web.id/\\
System se zakladni funkcionalitou a privetivym vzhledem.
Neni v cestine. Neni primarne urcet jako knihovni system, spis je to univerzalni
system na jakykoliv system, proto neni az tak efektivni a jeho nastavovani
by zabralo mnoho casu.

\section{Zaverem}
Evergreen a SLIMS jsou systemy, ktere poterbuji silne vyskolenou osobu,
aby se o system starala, narozdil od KOHY, ktera je intuitivnejsi a
pro nove uzivatele privetivejsi. Pri zachovani stejne, mozna i lepsi funkcionality.\\
\\

