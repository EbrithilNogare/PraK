\chapter{Vyber technologii}


\section{Frontend}
Vzhledem k rychle se menicim trendum v oblasti webovych technologii,
jsem se rozhodl jit cestou kterou vyvynul Facebook a jeho tym programatoru.
Jedna se o technologii \textbf{Single page aplication},
jez je implementovana v knihovne \textbf{React}.

\subsection{Single page aplication}
Single page aplication je technologie umoznujici vykresleni jine stranky,
bez nutnosti posilani requestu na server.
Uzivatel si pri prvnim spusteni webu stahne cely balicek webu a 
pri opetovnem nascteni vetsinou saha jen do sve cache.
Javascriptova knihovna (v tomto pripade React)
pote stranku prekresluje pri uzivatelske interakci.
V pripade nutnosti stazeni / posilani dat mezi serverem a uzivatelem
(napr. editace zaznamu, nebo nacteni existujiciho zaznamu)
se vola pouze request k API webove sluzby a telo requestu obsahuje pouze
uzitecne informace. 


\subsection{React}
Knihovna React je knihovna poskytujici single page aplication technologii.
Jedna se o dobre udrzovanou knihovnu, jez byla vyvinuta Facebookem, 
jakozto nahrada zastaraleho konceptu renderovani stranky na serveru.
Diky tomu servery nemuseli ztracet vykon s kazdou zmenou na strance a
vykon k renderovani se bere z PC uzivatele.
Jadro teto knihovny je velmi dobre optimalizovate a poskytuje i radu
debugovacich nastroju, coz je pro vetsi projekty nepostradatelna vyhoda.  


\subsection{Dalsi mozne technologie}
Velmi casto vykreslovani stranek probiha na serveru, se systemy jako jsou 
WordPress, psany PHP. Takovyto system je velmi dobre uzivatelsky privetivy,
ale z pohledu vykonu ma velmi obrovsky overhead. 
V pripade implementace knihovniho systemu by to znamenalo vykreslovat
celou stranku (hlavicku, telo i zapati) na serveru,
na druhe strane single page aplication nic nerenderuje,
pouze posle informaci o knize.


\section{Backend}
Mit single page aplikaci na frontendu znamena, ze na backendu musi existovat API,
od ktereho bude frontend cerpat data.
Navic zde potrebujeme i system pro staticke odesilani baliku cele webove stranky.
V ramci udrzitelnosti jsem se rozhodl vyuzit jazyk Javascript stejny jako pro frontend.
Express.js je knihovna ktera umoznuje komplexni spravu requestu a
stala se tudiz jasnou volbou.

\subsection{Express.js}
Express.js poskytuje odesilani statickych stranek (Reactiho baliku v nasem pripade),
custom requesty pro rozmanite API a take odesilani a lokalni ukladani statickych souboru,
jako obrazku, word i pdf dokumentu atd.

\subsection{MongoDB}
MongoDB je databazovy system typu non-sql.
Coz primarne znamena, ze data neuchovava v tabulkach, ale v tkz. schematech.
Coz ma mnoho vyhod, nejvetsi je, ze nekompletni zaznamy nezabiraji
svymi nevyplnenymi daty misto v DB a uklada se opravdu jen to zo je potreba.
Dalsi vyhodou, je styl ukladani dat a komunikace s DB.
Databaze si data uchovava ve formatu BSON (binarni JSON rozsireny o datove typy).
O data si aplikace zada pomoci query,
ktera je zcela odlisna od tech u sql-like databazi,
primarne se zde neposila query ve formatu string ale JSON,
diky cemuz napr. nenastane znama SQL injection.
Znovu ve formatu JSON pote data vraci aplikaci.

\subsection{Dalsi mozne technologie}
Diky oddeleni frontendu a backendu (narozdil napr. u WordPressu) je mozne
na backend nasadit temer cokoliv co umi posilat requesty.
Prikladem tomu muzou byt scripty v jazicich PHP, C\#, Python, nebo Perl.
Ale vzhledem k tomu, ze jednim z modulu bude neuronova sit na vyhledavani,
vybiral jsem mezi Pythonem a JavaScriptem, jakozto 2mi jazyky, ktere
maji velmi dobre knihovny pro praci s neuronovymi sitemi.

