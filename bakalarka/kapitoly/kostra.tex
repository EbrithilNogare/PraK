\chapter{Kostra}

\section{zadani}
co se od celeho projektu ocekava, jakou cast v tom ma ma prace

\section{existujici produkty}
existujici knihovni systemy jako KOHA, jejich vyhody a nevyhody

\section{vyber technologii}
ktere technologie byli pouzity, proc zrovna tyhle, v cem jsou lepsi, dat na ne odkazy

\section{diagram systemu}
obrazek toho co je s cim spojeno

\section{implementace backendu}
technologie a implementacni detaily
\subsection{propojeni s databazi}
jaka databaze byla zvolena a proc
\subsection{prava}
system overovani prav pro akce uzivatelu pri pouziti API

\section{implementace frontendu}
technologie a implementacni detaily
\subsection{uzivatelske prostredi a grafika}
zvoleny design a implementacni detaily
\subsection{zadavatko}
zadavaci system knihovniho systemu
\subsection{redakcni system}
popis a styl reseni
\subsection{lokalizace}
3 jazyky a jejich preklad

\section{moduly}
jak do systemu pridat modul
\subsection{modul hologram}
popis a funkcionalita

\section{provazani backendu a frontendu, API}
technologie a vyuzita reseni
\subsection{API}
dokumentace API, zadavatko, CMS i uzivatele
\subsection{rychlost}
popsat implemntacni detaily napomahajici k vyssi rychlosti

\section{instalace a spusteni}
manual vcetne linku na zdrojaky

\section{vysledny web}
screenshoty a vysledek prace

\section{vyuziti}
kde se system bude vyuzivat a aktualne vyuziva, pro koho je
